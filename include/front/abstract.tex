{
\thispagestyle{empty}
\vspace*{40pt}
\begin{center}
\color{primary}
\fontsize{27pt}{0pt}\bfseries\selectfont\alt Abstract
\vspace*{10pt}
\end{center}
}

\noindent
Il problema del Set-Covering (Set-Covering Problem, SCP) è un problema di ottimizzazione combinatoria che consiste nel
determinare la più piccola collezione di elementi in una famiglia di sottoinsiemi di un insieme, in modo che ciascun
elemento di quest'ultimo sia coperto dal almeno un sottoinsieme all'interno di tale collezione.

Questo problema, insieme alle varianti che ne derivano, ha numerose applicazioni pratiche in molteplici ambiti. Viene ad
esempio impiegato nell'ambito della logistica per ottimizzare la gestione e la distribuzione delle risorse. Trova spazio
anche all'interno delle compagnie aree, dove viene utilizzato per semplificare e ottimizzare la gestione degli arei e
dei turni di lavoro dei piloti.

Questo lavoro propone l'implementazione di un algoritmo basato sul metodo di Frank-Wolfe, con
l'obiettivo di ricercare un modo efficiente di risolvere il rilassamento lineare del SCP.

Nella prima parte vengono introdotte nozioni di carattere generale riguardanti l'ottimizzazione matematica e la
programmazione lineare [1]. Inoltre, vengono brevemente descritte l'intuizione alla base dell'algoritmo del simplesso e la
modalità con cui questo algoritmo viene impiegato per risolvere problemi di programmazione lineare. Successivamente
vengono presentate l'idea generale alla base dell'algoritmo di Frank-Wolfe e una formulazione matematica per il problema
del Set-Covering, con l'obiettivo di analizzare l'applicazione di tale algoritmo al rilassamento lineare.
Infine, viene sviluppata un'implementazione dell'algoritmo per risolvere diverse istanze del problema, accompagnata da
un confronto con l'algoritmo del simplesso applicato alle stesse istanze.
