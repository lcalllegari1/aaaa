\chapter{Nozioni Fondamentali}
L'obiettivo di  questo capitolo è quello di presentare i concetti di base che vengono utilizzati nel seguito di questo
lavoro e, più in generale, nell'ambito dell'ottimizzazione matematica.

\section{Introduzione}
Nel corso della sua esistenza, l’uomo ha sempre dovuto affrontare e risolvere una grande varietà di problemi. Con il
passare del tempo, le nostre capacità si sono evolute e gli strumenti a nostra disposizione sono migliorati,
permettendoci di gestire problemi di complessità sempre maggiore. Di conseguenza, oggi non ci accontentiamo più di
risolvere un problema trovando una soluzione qualsiasi, ma aspiriamo ad ottimizzare, cioè a identificare la soluzione
migliore possibile, sulla base di criteri specifici.

L'obiettivo di questo lavoro è quello di sviluppare un algoritmo risolutivo per il rilassamento
lineare del problema del set-covering, di cui verrà data una specifica successivamente, che sia in grado di risolvere
istanze del problema in modo efficiente. L'implementazione di tale algoritmo si basa sul metodo di Frank-Wolfe,
anch'esso approfondito nel seguito, e l'idea è quella di operare un confronto con l'algoritmo del simplesso.
